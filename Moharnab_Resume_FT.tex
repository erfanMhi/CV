%% The MIT License (MIT)
%%
%% Copyright (c) 2015 Daniil Belyakov
%%
%% Permission is hereby granted, free of charge, to any person obtaining a copy
%% of this software and associated documentation files (the "Software"), to deal
%% in the Software without restriction, including without limitation the rights
%% to use, copy, modify, merge, publish, distribute, sublicense, and/or sell
%% copies of the Software, and to permit persons to whom the Software is
%% furnished to do so, subject to the following conditions:
%%
%% The above copyright notice and this permission notice shall be included in all
%% copies or substantial portions of the Software.
%%
%% THE SOFTWARE IS PROVIDED "AS IS", WITHOUT WARRANTY OF ANY KIND, EXPRESS OR
%% IMPLIED, INCLUDING BUT NOT LIMITED TO THE WARRANTIES OF MERCHANTABILITY,
%% FITNESS FOR A PARTICULAR PURPOSE AND NONINFRINGEMENT. IN NO EVENT SHALL THE
%% AUTHORS OR COPYRIGHT HOLDERS BE LIABLE FOR ANY CLAIM, DAMAGES OR OTHER
%% LIABILITY, WHETHER IN AN ACTION OF CONTRACT, TORT OR OTHERWISE, ARISING FROM,
%% OUT OF OR IN CONNECTION WITH THE SOFTWARE OR THE USE OR OTHER DEALINGS IN THE
%% SOFTWARE.

% The font could be set to Windows-specific Calibri by using the 'calibri' option
\documentclass[]{mcdowellcv}

% For mathematical symbols
\usepackage{amsmath}
\usepackage{multicol}
\usepackage{blindtext}
\usepackage{hyperref}
\hypersetup{
    colorlinks=true,
    linkcolor=blue,
    filecolor=magenta,      
    urlcolor=cyan,
}

% Set applicant's personal data for header
\name{Erfan Miahi}
\address{Kamelia Apt., 8th alley, Bayani town, Rasht, Iran} 
\webpage{\href{https://erfanmhi.github.io}{erfanmhi.github.io}}
\contacts{+98 936 129 2690 \linebreak \href{mailto:mhi.erfan1@gmail.com}{mhi.erfan1@gmail.com}}
\github{\href{https://github.com/erfanMhi}{github.com/erfanMhi}}


\begin{document}

	% Print the header
	\makeheader
	
	% Print the content

    \begin{cvsection}{Education}
    	\begin{cvsubsection}{University of Guilan}{Rasht, Iran}{Sep 2015 -- Aug 2019}
    	    \setlength{\columnsep}{-2.1in}
    		\begin{itemize}
    			\item \textbf{B.Sc in Computer Engineering, GPA: 19.43/20 }
    			\item \textbf{Ranked 1\textsuperscript{st} in class}
    		\end{itemize}
    	\end{cvsubsection}
    	
    	\begin{cvsubsection}{Shahid Babayi High School}{Qazvin, Iran}{Sep 2014 -- May 2015}
    	    \setlength{\columnsep}{-2.1in}
    		\begin{itemize}
    			\item \textbf{Pre-University Diploma in Mathematics and Physics Discipline,  GPA: 18.67/20}
    			\item A branch of the National Organization for Development of Exceptional Talents (NODET) 
    		\end{itemize}
    	\end{cvsubsection}
    	
    	\begin{cvsubsection}{Shahid Babayi High School}{Qazvin, Iran}{Sep 2011 -- May 2014}
    	    \setlength{\columnsep}{-2.1in}
    		\begin{itemize}
    			\item \textbf{Diploma in Mathematics and Physics Discipline, Average GPA of 9 to 11 Grades: 18.06/20}
    			\item A branch of the National Organization for Development of Exceptional Talents (NODET) 
    		\end{itemize}
    	\end{cvsubsection}
    \end{cvsection}
    
    
% 	\begin{cvsection}{ENGLISH LANGUAGE PROFICIENCY}
%     	\begin{cvsubsection}{IELTS Academic Test}{}{July 2019}
%     	Overall Test Score: \textbf{7.0}
%     	\end{cvsubsection}
%     \end{cvsection}
    \begin{cvsection}{Research Interests}
        \begin{cvsubsection}{}{}{}
            \begin{multicols}{2}
                \begin{itemize}
                    \item Deep Reinforcement Learning
                    \item Machine Learning
                    \item Reinforcement Learning
                    \item Robotics
                    \item Computer Vision
                    \item Optimization
                    \item Deep Learning
                    \item Artificial Intelligence
                    
                    
                \end{itemize}
            \end{multicols}
        \end{cvsubsection}
    \end{cvsection}
    
    \begin{cvsection}{Skills and Expertise}
    	\begin{cvsubsection}{}{}{}
    	\end{cvsubsection}
    	\begin{itemize}
    	    \item \textbf{Programming Languages}
    	    \begin{itemize}
    	        \item \textbf{Python:} It is my main programming language. I have used it in various fields such as Backend Programming, Machine Learning, and Deep Learning for approximately four years.
    	        \item \textbf{Matlab:} It has been practiced for solving assignments of the Machine Learning course by Dr. Andrew Ng and the Principles of Computer Vision course at my university. 
    	        \item \textbf{C++:} I implemented several modules for the \textbf{Robotic Operating System (ROS)} using C++ during my participation in the Partner Robot Challenge. Moreover, I adopted it for teaching algorithms as a teaching assistant and solving assignments of Principles of Computer \& Programming class. 
    	        \item \textbf{Java:} I mainly used Java for implementation of assignments and projects of my university courses which were involved in concepts, namely \textbf{Object-oriented Programming}, \textbf{Multi-threading}, and \textbf{Socket Programming}, and frameworks, such as \textbf{Javafx}. Furthermore, when I was the teaching assistant of Advanced Programming and Data Structures courses, I employed it for teaching algorithms, data structures, and various concepts in Object-oriented Programming.
    	        \item \textbf{Other languages}: TypeScript, JavaScript, Assembly, VHDL.
    	    \end{itemize}
    	    \item{\textbf{Reinforcement Learning:}} Since 2018, I have been studying this topic. In detail, I have learned theoretical and practical aspects of this framework through a book entitled \textbf{Reinforcement Learning: An Introduction}, \href{http://www0.cs.ucl.ac.uk/staff/d.silver/web/Teaching.html}{\textbf{UCL Course on RL}}, and \textbf{RL Specialization in Coursera (offered by AMII)}.
    	    \item{\textbf{Mathematics:}} I achieved the full score in almost all of the mathematical courses which I have attended at my university. Additionally, I have expanded my knowledge about Linear Algebra and Statistics through \textbf{Introduction to Probability - The Science of Uncertainty course from Edx} and \textbf{Linear Algebra Course by Gilbert Strang on YouTube} as well as reading a lot of blogs, documents, and research papers.
    	    \item \textbf{Deep Learning:} Since 2017, I have been working with these algorithms, and I am expert in \textbf{Keras} and \textbf{Tensorflow} and intermediate in \textbf{Pytorch framework}. Besides, it is good to mention that I am proficient in \textbf{Hyperparameter Optimization}, especially \textbf{Neural Architecture Search (NAS)} technique.
    	    \item \textbf{Computer Vision:} I am expert in \textbf{Convolutional Neural Network architectures}, intermediate in \textbf{Geometric Deep Learning}, familiar with \textbf{OpenCV}, and a couple of Computer Vision tools in Matlab.
    	    \item \textbf{Machine Learning:} I am expert in \textbf{Scikit-learn} library and familiar with most of the ML algorithms like \textbf{Decision Trees}, \textbf{Random Forest}, and \textbf{SVM}.
    	    \item \textbf{Robot Operating System (ROS):} I, mainly, have used it in an international robotics competition.
    	    \item \textbf{Natural Language Processing:} I am intermediate in using deep learning for NLP and familiar with \textbf{NLTK} and \textbf{Spacy} libraries.
            \item \textbf{Databases:} I am expert in \textbf{Mysql}, intermediate in \textbf{Neo4j} and familiar with \textbf{MongoDB}.
            \item \textbf{Server Side Programming:} I have a good experience in practical usage of Python web frameworks such as \textbf{Django} and \textbf{Flask}.
            \item \textbf{Front-end Programming:} I have a good background in \textbf{Angular 5}, \textbf{HTML}, \textbf{CSS}, \textbf{Javascript}, and \textbf{Typescript}.
    	    \item \textbf{Productivity Software:} I have well-developed skills in usage of \LaTeX\ ,  Microsoft Office (Word, PowerPoint, Excel), and \textbf{Jupyter Notebook}.
    	    \item \textbf{Operating System:} I am proficient in GNU/Linux (3.5 years of usage) and Microsoft Windows systems. My favorite Linux distribution is Ubuntu.
    	    \end{itemize}
    
    \end{cvsection}
    
    \begin{cvsection}{Research Projects and Experiences}
    
    	\begin{cvsubsection}{Research Assistant}{University of Guilan}{2017 -- 2018}
    	    \setlength{\columnsep}{-2.1in}
    	    \textbf{Supervisor: Dr. M. Shakeri\newline Title: An efficient quantum-inspired genetic algorithm for the feature selection problem} \newline
            This research project was about designing a hybrid quantum-inspired genetic algorithm to select informative features for a neural network. In this project, I learned both the theory and implementation of metaheuristics and ML algorithms, such as k-means clustering and neural networks, as well as statistical analysis methods.
    	\end{cvsubsection}
    	
    	\begin{cvsubsection}{Research Assistant}{University of Guilan}{2017 -- 2019}
    	    \setlength{\columnsep}{-2.1in}
    	    \textbf{Supervisors: Dr. S. A. Mirroshandel \& Alexis Nasr\newline Title: Genetic neural architecture search for automatic assessment of human sperm image} \newline
            In this project, we proposed a Neural Architecture Search algorithm that adopted a special genetic algorithm as a neural architecture optimizer. During this process, I have learned not only the implementation and theory of various DL algorithms and hyperparameter optimization but also to write academic research papers.
    	\end{cvsubsection}
    	
    	\begin{cvsubsection}{Research Assistant}{University of Guilan}{2018 -- 2019}
    	    \setlength{\columnsep}{-2.1in}
    	    \textbf{Supervisor: Dr. S. A. Mirroshandel\newline Title: Benchmarking metaheuristics for neural architecture search } \newline
            In this research project, we benchmarked four metaheuristic algorithms- namely Harmony Search (HS), Artificial Bee Colony (ABC), Ant Colony Optimization (ACO), and Particle Swarm Optimization (PSO)- to automate the design of convolutional neural networks on a medical imaging dataset.
    	\end{cvsubsection}
    	
    	
    	\begin{cvsubsection}{Research Assistant}{University of Guilan}{2019 -- present}
    	    \setlength{\columnsep}{-2.1in}
    	    \textbf{Supervisor: Dr. M. Shakeri\newline Title: The blessing of cooperation in evolution for efficient and scalable transfer optimization} \newline
            In this project, a new evolutionary transfer optimization algorithm was introduced. We have evaluated this algorithm on combinatorial optimization problems, such as Knapsack, and double pole balancing problem. In detail, it consists of two parallelized evolutionary algorithms. The first one acts as an optimizer for the target task, and the other one was responsible for transferring knowledge from source tasks to the first one. At this moment, all the experimental results of this project have been gathered, and we are prepared to write a research paper.
    	\end{cvsubsection}
    	
    	
    	\begin{cvsubsection}{Research Assistant}{University of Guilan}{2019 -- present}
    	    \setlength{\columnsep}{-2.1in}
    	    \textbf{Supervisor: Dr. S. A. Mirroshandel\newline Title: Ensemble Transfer Learning for Sperm Morphology Analysis} \newline
            In this research project, we employed an ensemble of pre-trained convolutional neural networks to improve the accuracy of sperm abnormality detection. Currently, all experiments are done, and we are in the process of writing the research paper.

    	\end{cvsubsection}
    	
	\end{cvsection}
    
    \begin{cvsection}{Working Papers}
    	\begin{cvsubsection}{}{}{}
    	   \begin{itemize}
        	 \item E. Miahi, S. A. Mirroshandel, A. Nasr. \textbf{Genetic neural architecture search for automatic assessment of human sperm image} (Under review) \href{https://arxiv.org/abs/1909.09432}{Pre-print version on Arxiv}
        	 \item E. Miahi, M. Shakeri \textbf{The blessing of cooperation in evolution for efficient and scalable transfer optimization} (In preparation)
        % 	 \item E. Miahi, M. Shakeri  \textbf{An efficient quantum-inspired genetic algorithm for the feature selection problem} (In preparation)
        	 \item S. A. Mirroshandel, A. Abbasi, E. Miahi \textbf{Multi-task Transfer Learning for Sperm Morphology Analysis} (In preparation)
        	\end{itemize}
         	 
    	\end{cvsubsection}

    \end{cvsection}
    
    
    \begin{cvsection}{Honors and Awards}
    	\begin{cvsubsection}{}{}{}	
    		\begin{itemize}
    		    \item \textbf{Abrishamchian Reward (40,000,000 Rial = \$1,241.77 CAD)} \newline
                I obtained this reward because of my outstanding accomplishments among all students of the University of Guilan.
                It is good to mention that only 11 students among approximately 17,000 students of the University of Guilan achieved this reward, and just two of them had a bachelor's degree (\href{https://www.dropbox.com/s/avfvpvrg5pftjuc/Abrishamchian\%27s\%20Reward.png?dl=0}{certification}).
    			\item \textbf{World Robot Challenge 2018 (WRC2018)} \newline
                My team and I participated in the international World Robot Challenge 2018 (WRC2018), and we ranked 6th in the Partner Robot challenge (Virtual Space) category which was held in Japon. 126 teams from 23 countries participated in this competition (\href{https://www.dropbox.com/s/bp9kqpk0mmomlhz/World\%20Robot\%20Summit\%20Certification.jpg?dl=0}{certification}).
    			\item \textbf{Ranked 1\textsuperscript{st} in class} \newline
    			I had the highest GPA among my classmates (\href{https://www.dropbox.com/s/auc8cztccy4n15u/Ranked\%20First.jpg?dl=0}{certification}).
    			\item \textbf{Top Researcher in Computer Engineering Group} \newline
                I was announced as the top researcher in the Computer Engineering Group of the University of Guilan (\href{https://www.dropbox.com/s/sf187sfu1cetzzw/Top\%20Researcher\%20in\%20Computer\%20Engineering\%20Group\%20Certification.jpg?dl=0}{certification}).
    			\item \textbf{Full Scholarship, B.Sc, University of Guilan} \newline
                I was selected through a highly competitive national entrance exam. In this exam, the selected students should have a minimum rank of up to 1\%,  and it has approximately a total of 500,000 applicants.
                \item \textbf{National Organization for Development of Exceptional Talents (NODET)} \newline
                Recognized as a talented student in the entrance exam of NODET among Qazvin students for middle and high school.
                \item \textbf{ACM Honorable Mention} \newline
              I have participated in the \textbf{ACM International Collegiate Programming Contest (ICPC)} which was held in the Sharif University of Technology in 2016 (\href{https://www.dropbox.com/s/t3nfzaavpyrf8os/ACM\%20certification.pdf?dl=0}{certification}).
    		\end{itemize}
    	\end{cvsubsection}
    \end{cvsection}
    
    
	\begin{cvsection}{Certifications}
    		\begin{cvsubsection}{Deep Learning Specialization}{Coursera, by Andrew Ng}{Oct 2018}
                Deep Learning Specialization consists of five different courses:
    			\begin{itemize}
                \item Neural Networks and Deep Learning (\href{https://www.coursera.org/account/accomplishments/certificate/Q466THNX9825}{certification})
                \item Improving Deep Neural Networks: Hyperparameter tuning Regularization and Optimization (\href{https://www.coursera.org/account/accomplishments/verify/BKWJS632W3V8}{certification})
    			\item Structuring Machine Learning Projects (\href{https://www.coursera.org/account/accomplishments/certificate/UEF3FJTFZSFD}{certification})
    			\item Convolutional Neural Networks (\href{https://www.coursera.org/account/accomplishments/certificate/MEWWQGYXJX2M}{certification})
    			\item Sequence Models (\href{https://www.coursera.org/account/accomplishments/certificate/RDANP785FVS6}{certification})
    			\end{itemize}
    			You can find the certification \href{https://www.coursera.org/account/accomplishments/specialization/certificate/XCRZ9PFYVLNH}{here}.
    		\end{cvsubsection}
    		
    		\begin{cvsubsection}{Machine Learning Course}{Coursera, by Andrew Ng}{Nov 2018}
                You can find the certification \href{https://www.coursera.org/account/accomplishments/verify/JN6PYQVTHEGB}{here}.
    		\end{cvsubsection}
    		
			\begin{cvsubsection}{Reinforcement Learning Course}{Coursera, \textbf{by University of Alberta}}{Oct 2019}
            You can find the certification \href{https://www.coursera.org/account/accomplishments/verify/XXEYSHY47YW9}{here}.
    		\end{cvsubsection}
    		
    		\begin{cvsubsection}{Data Science Course}{Coursera, by Christopher Brooks}{March 2019} 
    		    
                You can find the certification \href{https://www.coursera.org/account/accomplishments/verify/CMAYYDRR3XG4}{here}.
    		\end{cvsubsection}
    		
			\begin{cvsubsection}{Data Visualization Course}{Coursera, by Christopher Brooks}{Aug 2019}
            You can find the certification \href{https://www.coursera.org/account/accomplishments/verify/JWAWUXXNVNWX}{here}.
    		\end{cvsubsection}
    		
            % \newpage
    		\begin{cvsubsection}{Deep Learning Summer School 2018}{University of Tehran}{Aug 2018}
            	\begin{itemize}
            	    \item A three-day school, covering different areas of deep learning such as MLP, CNNs, RNNs and GANs with a hands-on using the Keras framework (\href{https://github.com/erfanMhi/Deep-Learning-Summer-School}{hands-on materials}).
                    \item Several lecturers were from \textbf{Deepmind}, \textbf{Stanford University}, \textbf{UC Berekly} and \textbf{EPFL} University.
            	\end{itemize}
                You can find the certification \href{https://www.dropbox.com/s/i6jnb73ydrzsjbr/Deeplearning\%20Summer\%20School.jpg?dl=0}{here}.
    		\end{cvsubsection}
    		\begin{cvsubsection}{Brain \& Cognition Summer School}{National Brain Mapping Laboratory}{Sep 2018}
                A one-week school, covering different areas of \textbf{Cognitive Science}, \textbf{Neuroscience}, \textbf{Computational Neuroscience} and how to use various tools for getting data from the brain.
                You can find the certification \href{https://www.dropbox.com/s/46kc445t36e2yzi/photo_2018-09-07_14-08-06.jpg?dl=0}{here}.
    		\end{cvsubsection}
    		
    		\begin{cvsubsection}{Grammar and Punctuation}{Coursera}{Sep 2019}
                 You can find the certification \href{https://www.coursera.org/account/accomplishments/verify/3SZUUC77CCV7}{here}.
            \end{cvsubsection}     
            \begin{cvsubsection}{Research for Essay Writing}{Coursera}{Sep 2019}
                You can find the certification \href{https://www.coursera.org/account/accomplishments/verify/KPHH92Z6ZEXY}{here}.
            \end{cvsubsection}    
    \end{cvsection}

	
    \begin{cvsection}{Teaching Assistant Experiences}
    
    	\begin{cvsubsection}{Teaching Assistant}{University of Guilan}{2019}
    	    \setlength{\columnsep}{-2.1in}
    	    \textbf{Course: Principles of Data Mining \newline Instructor: Dr. M. Shakeri} \newline
    	    My only responsibility was assessing the students' assignments.
    	\end{cvsubsection}
    	
    	\begin{cvsubsection}{Teaching Assistant}{University of Guilan}{2018}
    	    \setlength{\columnsep}{-2.1in}
    	    \textbf{Course: Principles of Computational Intelligence \newline Instructor: Dr. M. Shakeri} \newline
            I designed a comprehensive tutorial on metaheuristic algorithms with special animated visualization in Python and taught it in the classes which I held for the students. You can find the tutorial \href{https://github.com/Computational-Intelligence-Fall18/Computational-Intelligence-Tutorials}{here}. Besides, I planned and assessed the final project of the students, which you can find it \href{https://github.com/Computational-Intelligence-Fall18/Student-Classification-Assignment}{here}.
    	\end{cvsubsection}
    	
    	\begin{cvsubsection}{Teaching Assistant}{University of Guilan}{2018}
    	    \setlength{\columnsep}{-2.1in}
    	    \textbf{Course: Algorithm Design \newline Instructor: Dr. M. Shakeri} \newline
    	    My only responsibility was assessing the students' assignments.
    	\end{cvsubsection}
    	
    	\begin{cvsubsection}{Teaching Assistant}{University of Guilan}{2017}
    	    \setlength{\columnsep}{-2.1in}
    	    \textbf{Course: Data Structures \newline Instructor: Dr. S. A. Mirroshandel} \newline
        During weekly sessions that I had with students, I taught them to implement complex data structures and algorithms, such as priority queue and hash tables, in Java. Moreover, I designed and assessed students' assignments.

    
    	\end{cvsubsection}
    	

    	
    	\begin{cvsubsection}{Teaching Assistant}{University of Guilan}{2016}
    	    \setlength{\columnsep}{-2.1in}
    	    \textbf{Course: Principles of Computer \& Programming \newline Instructor: Dr. S. M. Shekarian} \newline
        During this course, I had weekly classes with students, and, in these classes, I taught them to solve various fundamental algorithmic problems using the C++ programming language. Furthermore, I designed and assessed their weekly assignments. Lastly, I assessed their final project, which was designed by Dr. Seyed Mohammadhossein Shekarian.
    	\end{cvsubsection}
    	
    	\begin{cvsubsection}{Teaching Assistant}{University of Guilan}{2016}
    	    \setlength{\columnsep}{-2.1in}
    	    \textbf{Course: Advanced Programming \newline Instructor: Dr. S. A. Mirroshandel} \newline
        During this course, I had weekly classes with students, and, in these classes, I taught them to design various projects using the Java programming language. Furthermore, I designed and assessed five comprehensive assignments and the final project, which were involved in concepts such as Socket Programming, GUI Design, etc.
    	\end{cvsubsection}
    
    
    \end{cvsection}
    
    


        % \newpage
        
	\begin{cvsection}{Leadership Experiences}
    		
    		\begin{cvsubsection}{Brain and Cognition Association}{University of Guilan}{Aug 2018 -- Sep 2019}
		        \textbf{Vice Chairman of the Modeling and Artificial Intelligence Committee} \newline
                I was educating students in the fields of Artificial Intelligence and Cognitive Science through lecturing, holding events, and mentorship. 
    		\end{cvsubsection}
    		
        	\begin{cvsubsection}{Dean at Rasht School of AI}{School of AI, Rasht Chapter}{Oct 2018 -- Oct 2019}
            I held five meetups in Rasht and lectured in four of them about topics such as Artificial Intelligence, Computational Neuroscience, Data Science, and Optimization algorithms. The materials of these meetups can be found \href{https://github.com/school-of-ai-rasht-chapter/Meetup-Materials}{here}. Furthermore, I designed a learning path for students who want to learn ML and Deep Learning. You can found it in \href{https://t.me/joinchat/GtdKmRJ2jXuzymyXm2RuQA}{this Telegram channel}. Moreover, I am currently mentoring several students who are motivated to pursue Computational Neuroscience and machine learning.
            \end{cvsubsection}
    
    \end{cvsection}
	\begin{cvsection}{Employment}

    		\begin{cvsubsection}{Front-end Developer}{Nila Software Group, Rasht}{Oct 2016 –- Nov 2017}
                I was working on a project entitled \href{https://memaraneh.com/}{Memaraneh}, which was a website for introducing decorations and selling appliances.
    		\end{cvsubsection}
    		
   
    		\begin{cvsubsection}{Researcher}{Guilan NLP Group, Rasht}{November 2019 –- Present}
    		    I am participating in research projects involving NLP and Computer Vision with Guilan NLP Group.
                You can find our website \href{https://nlp.guilan.ac.ir/}{here}.
    		\end{cvsubsection}
    
    \end{cvsection}
    
        \begin{cvsection}{Selected Accomplished Projects}
    	\begin{cvsubsection}{}{}{}
        	\begin{itemize}
        	   % \item \textbf{Genetic Neural Architecture Search}: A genetic algorithm is used to find the best convolutional neural network architecture for solving classification problem on an medical imbalanced dataset.
        	    \item \textbf{Computational Intelligence Tutorial}: A tutorial designed for beginners who want to understand and implement various metaheuristic algorithms in Python. It contains a lot of insightful visualizations for understanding the exploration-exploitation dilemma (\href{https://github.com/Computational-Intelligence-Fall18/Computational-Intelligence-Tutorials}{repo}).
        	    \item \textbf{Deep Learning For Natural Language Processing:} I have been solving the assignments of deep learning for natural language processing course by Stanford University (\href{https://github.com/erfanMhi/cs224n_Assignments}{repo}).
        	   % \item \textbf{Quantum-inspired Genetic Algorithm for Feature Selection}: A quantum-inspired genetic algorithm which is specially designed to solve feature selection problems in a hybrid manner.
        	    \item \textbf{Genetic Algorithms For Credit Scoring}: I implemented a Hybrid Genetic Algorithm for feature selection to solve the credit scoring problem in Python (\href{https://github.com/erfanMhi/Genetic-algorithms-for-credit-scoring}{repo}).
        	    \item \textbf{Quantum-inspired Genetic Algorithm for K-means clustering}: I have implemented a Quantum Genetic Algorithm for initializing the first k points of k-means clustering algorithm in Python (\href{https://github.com/erfanMhi/A-quantum-inspired-genetic-algorithm-for-k-means-clustering}{repo}).
        	   % \item \textbf{Digit Recognizer}: I created an Ipython notebook and I trained Different CNN architecture in MNIST-Digits dataset, provided by Kaggle, to see how well they work in comparison with each other (\href{https://github.com/erfanMhi/Digit_Recognizer}{repo}).
        	    \item \textbf{Persian Digit Recognizer}: It is a well-written Jupyter notebook for learning the implementation of a Convolutional Neural Network for recognizing Persian digits (\href{https://github.com/erfanMhi/Persian-Digits-Recognizer}{repo}).
        	    \item \textbf{Persian News Classification}: It was the final project of Principles of Language \& Speech Processing course (\href{https://github.com/Nikronic/NLP-Fall18-UOG/}{repo}).
        	    \item \textbf{Kdnuggets Data Mining}: I tackled assignments of Kdnuggets data mining course in Jupyter notebook using Python (\href{https://github.com/erfanMhi/kdnuggets-data-mining-course}{repo}).
        	    \item \textbf{Event Scheduling Website}: I created a website for scheduling events using Django framework (\href{https://github.com/erfanMhi/School-Of-AI-Rasht-Chapter-Website}{repo}).
        	    \item \textbf{Udacity Deep Learning:} I solved assignments of the Deep Learning course by Udacity
        	    (\href{https://github.com/erfanMhi/Udacity-Deep-Learning-Assignment}{repo}).
        	    \item \textbf{Music Recommender}: It was implemented with Python, and it was the final project of the Artificial Intelligence \& Expert Systems course.
        	   % \item \textbf{Genetic Neural Architecture Search}: A Genetic Algorithm is used to find the best convolutional neural network for solving classification problem on an unbalanced dataset.
        	   % \item \textbf{Deep Learning For Natural Language Processing:} I solved assignments of deep learning for natural language processing course by Stanford University (\href{https://github.com/erfanMhi/cs224n_Assignments}{repo}).
        	   % \item \textbf{Persian News Classification}: Final project of Natural language processing course by University of Guilan (\href{https://github.com/Nikronic/NLP-Fall18-UOG/}{repo}).
        	   % \item \textbf{Kdnuggets Data Mining}: Solved assignments of Kdnuggets data mining course in jupyter notebook using python (\href{https://github.com/erfanMhi/kdnuggets-data-mining-course}{repo}).
        	   % \item \textbf{Genetic Algorithms For Credit Scoring}: I implemented a Hybrid Genetic Algorithm for feature selection to solve the credit scoring problem in Python (\href{https://github.com/erfanMhi/Genetic-algorithms-for-credit-scoring}{repo}).
        	   % \item \textbf{Quantum Genetic Algorithm for K-means clustering}: I implemented a Quantum Genetic Algorithm for initializing the first k points for k-means clustering in Python (\href{https://github.com/erfanMhi/A-quantum-inspired-genetic-algorithm-for-k-means-clustering}{repo}).
        	   % \item \textbf{Digit Recognizer}: I created an Ipython notebook and I trained Different CNN architecture in MNIST-Digits dataset, provided by Kaggle, to see how well they work in comparison with each other (\href{https://github.com/erfanMhi/Digit_Recognizer}{repo}).
        	   % \item \textbf{Persian Digit Recognizer}: A well-written Jupyter notebook for learning how you could implement a Convolutional Neural Network for recognizing Persian digits (\href{https://github.com/erfanMhi/Persian-Digits-Recognizer}{repo}).
        	   % \item \textbf{Event Scheduling Website}: I created a website for scheduling events using Django framework (\href{https://github.com/erfanMhi/School-Of-AI-Rasht-Chapter-Website}{repo}).
        	   % \item \textbf{Udacity Deep Learning:} I solved assignments of deep learning course by Udacity
        	   % (\href{https://github.com/erfanMhi/Udacity-Deep-Learning-Assignment}{repo}).
        	   % \item \textbf{Music Recommender}: Implemented with Python and it was the final project of Artificial Intelligence course.
        	   % \item \textbf{Memaraneh}: It was website design projects and I used Angular 5 for implementing the project.
 
        	\end{itemize}
	    \end{cvsubsection}

    \end{cvsection}
    
    
    \begin{cvsection}{Hobbies}
    	\begin{cvsubsection}{}{}{}
    	
    	\begin{itemize}
    	    \item \textbf{Learning New Things:} I am passionate about learning itself. Therefore, I spent most of my free time learning more about Philosophy, Physics, Neuroscience, Psychology, and Mathematics through MOOCS and books. Moreover, I have a Telegram channel which I use it to share educational content about aforementioned topics. It can be found \href{https://t.me/my_personal_expriences}{here}.
    	    \item \textbf{Sport:} I am passionate about different types of sports and proficient in \textbf{Parkour}. You can find the video of my Parkour movements in \href{https://www.instagram.com/erfan_mhi/}{my Instagram page}.
    	    \item \textbf{Music:} I have had a deep enthusiasm for listening to different music genres, and my favorite musical instrument is Piano.
	    \end{itemize}
	    
    	\end{cvsubsection}
    \end{cvsection}
	
    \begin{cvsection}{Languages}
    	\begin{cvsubsection}{}{}{}
    	
    	\begin{itemize}
    	    \item \textbf{Persian:} Native
    	    \item \textbf{English:} Fluent
	    \end{itemize}
	    
    	\end{cvsubsection}
    \end{cvsection}
    
% 	\newpage

	 \begin{cvsection}{References}
	 
 		\begin{cvsubsection}{Dr. Mojtaba Shakeri}{Singapore Institute of Manufacturing Technology \& University of Guilan}{Singapore \& Rasht, Iran}
		    Scientist \& Assistant Professor \newline
			Email: \href{mailto:Mojtaba\_Shakeri@simtech.a-star.edu.sg}{Mojtaba\_Shakeri@simtech.a-star.edu.sg} \newline
			Home: \href{https://staff.guilan.ac.ir/mshakeri/index.php?a=0\&lg=1}{https://staff.guilan.ac.ir/mshakeri/index.php?a=0\&lg=1}
    	\end{cvsubsection}
		\begin{cvsubsection}{Dr. Seyed Abolghasem Mirroshandel}{University of Guilan}{Rasht, Iran}
		    Head of Computer Engineering Department \& Associate Professor \newline
			Email: \href{mailto:mirroshandel@guilan.ac.ir}{mirroshandel@guilan.ac.ir} \newline
			Home: \href{https://staff.guilan.ac.ir/mirroshandel/?lg=1}{https://staff.guilan.ac.ir/mirroshandel/?lg=1}

    	\end{cvsubsection}
    	
		\begin{cvsubsection}{Prof. Alexis Nasr}{Université Aix Marseille}{Marseille, France}
		    Professor \newline
			Email: \href{mailto:Alexis.Nasr@lis-lab.fr}{Alexis.Nasr@lis-lab.fr} \newline
			Home: \href{http://pageperso.lif.univ-mrs.fr/~alexis.nasr/}{http://pageperso.lif.univ-mrs.fr/~alexis.nasr/}
    	\end{cvsubsection}
    	


    \end{cvsection}
	
\end{document}
